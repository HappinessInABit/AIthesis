%%%%%%%%%%%%%%%%%%%%%%%%%%%%%%%%%%%%%%%%%
% Journal Article
% LaTeX Template
% Version 1.4 (15/5/16)
%
% This template has been downloaded from:
% http://www.LaTeXTemplates.com
%
% Original author:
% Frits Wenneker (http://www.howtotex.com) with extensive modifications by
% Vel (vel@LaTeXTemplates.com)
%
% License:
% CC BY-NC-SA 3.0 (http://creativecommons.org/licenses/by-nc-sa/3.0/)
%
%%%%%%%%%%%%%%%%%%%%%%%%%%%%%%%%%%%%%%%%%

%----------------------------------------------------------------------------------------
%	PACKAGES AND OTHER DOCUMENT CONFIGURATIONS
%----------------------------------------------------------------------------------------

\documentclass[twoside]{article}

\usepackage{blindtext} % Package to generate dummy text throughout this template

\usepackage[sc]{mathpazo} % Use the Palatino font
\usepackage[T1]{fontenc} % Use 8-bit encoding that has 256 glyphs
\linespread{1.05} % Line spacing - Palatino needs more space between lines
\usepackage{microtype} % Slightly tweak font spacing for aesthetics

\usepackage[english]{babel} % Language hyphenation and typographical rules

\usepackage[hmarginratio=1:1,top=32mm,columnsep=20pt]{geometry} % Document margins
\usepackage[hang, small,labelfont=bf,up,textfont=it,up]{caption} % Custom captions under/above floats in tables or figures
\usepackage{booktabs} % Horizontal rules in tables

\usepackage{lettrine} % The lettrine is the first enlarged letter at the beginning of the text

\usepackage{enumitem} % Customized lists
\setlist[itemize]{noitemsep} % Make itemize lists more compact

\usepackage{abstract} % Allows abstract customization
\renewcommand{\abstractnamefont}{\normalfont\bfseries} % Set the "Abstract" text to bold
\renewcommand{\abstracttextfont}{\normalfont\small\itshape} % Set the abstract itself to small italic text

\usepackage{titlesec} % Allows customization of titles
\renewcommand\thesection{\Roman{section}} % Roman numerals for the sections
\renewcommand\thesubsection{\roman{subsection}} % roman numerals for subsections
\titleformat{\section}[block]{\large\scshape\centering}{\thesection.}{1em}{} % Change the look of the section titles
\titleformat{\subsection}[block]{\large}{\thesubsection.}{1em}{} % Change the look of the section titles

\usepackage{fancyhdr} % Headers and footers
\pagestyle{fancy} % All pages have headers and footers
\fancyhead{} % Blank out the default header
\fancyfoot{} % Blank out the default footer
\fancyhead[C]{Running title $\bullet$ May 2016 $\bullet$ Vol. XXI, No. 1} % Custom header text
\fancyfoot[RO,LE]{\thepage} % Custom footer text

\usepackage{titling} % Customizing the title section

\usepackage{hyperref} % For hyperlinks in the PDF

%----------------------------------------------------------------------------------------
%	TITLE SECTION
%----------------------------------------------------------------------------------------

\setlength{\droptitle}{-4\baselineskip} % Move the title up

\pretitle{\begin{center}\Huge\bfseries} % Article title formatting
\posttitle{\end{center}} % Article title closing formatting
\title{Your fancy paperrrr title} % Article title
\author{%
\textsc{John Smith} \\[1ex] % Your name
\normalsize Your fancy institution \\ % Your institution
\normalsize \href{mailto:john@smith.com}{john@smith.com} % Your email address
%\and % Uncomment if 2 authors are required, duplicate these 4 lines if more
%\textsc{Jane Smith}\thanks{Corresponding author} \\[1ex] % Second author's name
%\normalsize University of Utah \\ % Second author's institution
%\normalsize \href{mailto:jane@smith.com}{jane@smith.com} % Second author's email address
}
\date{\today} % Leave empty to omit a date
\renewcommand{\maketitlehookd}{%
\begin{abstract}
\noindent \blindtext
\end{abstract}
}

%----------------------------------------------------------------------------------------

\begin{document}

% Print the title
\maketitle

%----------------------------------------------------------------------------------------
%	ARTICLE CONTENTS
%----------------------------------------------------------------------------------------

\section{Introduction}
\paragraph{General}
Mutliple disorders can disrupt the neuromuscular channels our brain uses for motor control of our bodies. Amyotrophic lateral sclerosis (ALS), cerebral palsy and muscular dystrophies are a few of such disorders that lead to severe motor impairment and communication disabilities [SOURCE]. Brain computer interfacing (BCIs) is a highly interdisciplinary field with the goal to build a direct external communication channel between the human brain and artifical devices. Great interest in  BCI technology was generated in 1999 when a study demonstrated direct control of a robotic arm via ensembles of cortical neurons \cite{Chapin:1999}. Since, proof of concept studies have shown BCI control over computer cursers, computer spelling systems, robotic arms and wheelchairs [SOURCES 3,31 IN CITATION 1]. Its potential to improve of the quality of life of neurologically impaired patients drives research in BCI techonology.
%Patients suffering of ALS, cerebral palsy are amoungst %excitement in
% Quality of life

\paragraph{EEG}
Amongst the types of BCIs in developement, electroencephalogy (EEG) is one of the most studied methods of electrophysiological recodings. It is non-invasive system that does not expose patients to the risks of brain surgery. HOW ITS downloaded
EEG can detect modulations in brain activity that correlate

\paragraph{P300}
Event related response P300
informs stimulus selection
machine learning algorithms have been trained to recognized EEG P300 patterns [SOURCE 26 IN CITATION 1].

\paragraph{Thesis need}
Work addressing THIS ISSUE, will enable better control and broad use of prostetics?


\blindtext

\blindtext

\section{Definitions}

\blindtext

\blindtext
\section{Discussion}
EEG
  have limited capacity in transfer rate [Lebedev MA et al 2006]
  spatial and temporal resolution is is limited [Lebedev MA et al 2006]
    activity is captured from multiple cortical areas
    signal going through brain tissue, bone and skin damps resolution


%------------------------------------------------

\section{Methods}

Some random text \footnote{Example footnote}, and more random text:

\begin{itemize}
\item Item 1 \cite{Figueredo:2009dg}
\item Item 2
\item Item 3
\end{itemize}



%----------------------------------------------------------------------------------------
%	REFERENCE LIST
%----------------------------------------------------------------------------------------

\begin{thebibliography}{99} % Bibliography - this is intentionally simple in this template

\bibitem[Figueredo and Wolf, 2009]{Figueredo:2009dg}
Figueredo, A.~J. and Wolf, P. S.~A. (2009).
\newblock Assortative pairing and life history strategy - a cross-cultural
  study.
\newblock {\em Human Nature}, 20:317--330.

\bibitem[]{Chapin:1999}
Chapin, John K., et al. (1999).
\newblock Real-time control of a robot arm using simultaneously recorded neurons in the motor cortex.
\newblock {\em Nature neuroscience}, 2.7 : 664-670.

\end{thebibliography}

%----------------------------------------------------------------------------------------

\end{document}
